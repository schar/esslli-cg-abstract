%\begin{pre}
\documentclass[11pt,twocolumn,a4paper]{bqhatevwr}
\usepackage{etex,amsmath,mathtools,newtxtext,newtxmath,expex,natbib,tikz-qtree,bookmark,tree-dvips}%
\usepackage[margin=1in]{geometry}
\usepackage{bussproofs}\EnableBpAbbreviations
\newcommand{\pruf}[1]{\def\ScoreOverhang{0pt}\def\defaultHypSeparation{\hskip .25em}#1\DisplayProof}%
\newcommand{\lab}[1]{\RightLabel{\scriptsize #1}}
\usepackage{enumitem}
\setlist[itemize]{leftmargin=0pt}
\def\labelitemi{} 
\lingset{exskip=.5em}
\delimitershortfall=-1pt
\bibliographystyle{bqhatevwr}
\author[D.~Bumford \& S.~Charlow]{\spauthor{Dylan Bumford \\ \institute{NYU}} \AND
  \spauthor{Simon Charlow \\ \institute{Rutgers}} \AND
}
\title{Monadic dynamic semantics}
%\end{pre}
\begin{document}
%\maketitle

\section{Outline}
%\begin{sec}
	\textbf{C}ontinuized \textbf{C}ombinatory \textbf{C}ategorial \textbf{G}rammars bring scope-takers such as quantificational DPs into the compositional fold by allowing any expression to take scope (e.g.~\citealt{ShanBarker:2006, BarkerShan:2008, BarkerShan:2014}). Roughly, this is accomplished by adding three combinators to an applicative categorial grammar: \bsf{lift} turns any expression into a function on its linguistic context, or \emph{continuation}; \bsf{scope} combines two scope-takers to yield a third; and \bsf{triv} is a trivial scope argument used to conclude derivations and delimit a context of evaluation.%

	Though CCCGs offer robust accounts of in-scope binding (ibid.), the treatment of dynamic binding (i.e.~cross-sentential and donkey anaphora) proposed in \citealt{BarkerShan:2008} over-generates (\citealt{Charlow:2010}; see \citealt{BarkerShan:2014} for in-depth discussion). Conversely, the related theory of \citealt{Groote:2006} (which relies on continuations, but is not a CCG), while treating dynamic binding, does not offer a general account of scope-taking. In addition, both \citealt{BarkerShan:2008} and \citealt{Groote:2006} treat dynamic binding by indefinites, but not by expressions such as \emph{exactly one linguist}, which differs in important ways (cf.~\citealt{KampReyle:1993}). %
	
	%Three combinators, \bsf{lift}, \bsf{triv}, and \bsf{scope}, form the backbone of the grammar, allowing scope-takers to interact with their linguistic context. In addition, a \bsf{bind} shifter is introduced to facilitate quantificational binding. \citealt{Groote:2001}.%
	
	This paper presents an explicit CCCG account of dynamic binding inspired by the category-theoretic notion of a monad (e.g.~\citealt{Moggi:1989, Wadler:1992, Wadler:1994, Wadler:1995, Shan:2002}). Monads are a device for enriching a purely applicative grammar with \emph{side effects}---roughly, things that happen in the course of semantic composition besides functional application on values (\citealt{Shan:2005}). Concretely, a monad determines a pair of combinators \ab{\eta,\,\star}. I propose replacing the \bsf{lift} and \bsf{triv} of CCCGs, respectively, with $\eta$ and $\star$. This has the automatic effect of producing a CCCG that countenances side effects.%
	
	 Any side effects regime can be grafted onto a continuized CCG in this way. I provide a general technique for integrating a monadic approach to side effects with continuations-based approaches to scope in CCG. We relate our approach to the ContT monad transformer (\citealt{Liangetal}). Offers a type-theoretic way to track effects, integrate them into a well-developed CCG framework for scope-taking. I illustrate with a single in-depth case study, the case of dynamic binding. This involves find a monad for state and nondeterminism, positing some lexical entries, and \emph{nothing else}.%
	
	Therefore, these results are of interest both for the categorial grammarian interested in donkey anaphora and scope-taking, as well as more generally. Any side-effects regime a semanticist thinks is motivated can be accommodated along these lines. Further, because of the inherent modularity, adding side effects necessitates neither fiddling with the basic compositional machinery, nor messing with lexical items which don't exploit side effects.%
	
	The standard continuations-based perspective of \citealt{Barker:2002, ShanBarker:2006, BarkerShan:2014} is an instantiation of a more general perspective.%
%\end{sec}

\section{Continuations}
%\begin{sec}
	Figure \ref{fig1}.
%\end{sec}

\section{Adding side effects}
%\begin{sec}
	Standard continuized grammar is, simplifying somewhat, three combinators: \bsf{lift}, \bsf{triv}, and \bsf{scope}. Figure \ref{fig1}. %
	
	Continuized type constructor. Agnostic about directionality. Combined with direction-sensitive mode of combination. See below. $\textsf{K}\,a\,r$ is a meaning which functions as something of type $a$ in a context of type $r$ (the `result type'). For example, extensional generalized quantifiers have type $\textsf{K}\,e\,t$.%
	\[\textsf{K}\,a\,r \Coloneqq (a \ra r) \ra r\]
	\begin{figure*}
		{\small\[\begin{array}{@{}c@{}}
		\begin{array}{c@{\hspace{2em}}c@{\hspace{2em}}c@{\hspace{2em}}c}%
			\pruf{%
			\AXC{$\Gamma \vdash f : b/a~~$}
			\AXC{$~~\Delta \vdash e : a$}
			\lab{$/$}
			\BIC{$\Gamma \cdot \Delta \vdash f\,e : b$}
			}
			&
			\pruf{%
			\AXC{$\Delta \vdash e : a~~$}
			\AXC{$~~\Gamma \vdash f : a \backslash b$}
			\lab{$\backslash$}
			\BIC{$\Delta \cdot \Gamma \vdash f\,e : b$}
			}
			&
			\pruf{%
			\AXC{$\Delta \vdash m : \textsf{K}\,(b/a)\,r~~$}
			\AXC{$~~\Gamma \vdash n : \textsf{K}\,a\,r$}
			\lab{$\sslash$}
			\BIC{$\Delta \cdot \Gamma \vdash \bsf{S}\,m\,n : \textsf{K}\,b\,r$}
			}
		\end{array}
		\\\\
		\begin{array}{c@{\hspace{2em}}c}
				\pruf{%
				\AXC{}
				\lab{\bsf{triv}}
				\UIC{$\varepsilon \vdash \lambda x.\,x : a \ra a$}
				}
				&
				\pruf{%
				\AXC{}
				\lab{\bsf{lift}}
				\UIC{$\varepsilon \vdash \lambda xk.\,k\,x:  a \ra \textsf{K}\,a\,r$}%
				}
		\end{array}
		\\[-1em]
		\end{array}\]}
		\caption{Continuized CCG without side effects, fixing a result type $r$.}%
		\label{fig1}
	\end{figure*}
	\begin{figure*}
		{\small\[\begin{array}{c}
		\begin{array}{c@{\hspace{2em}}c@{\hspace{2em}}c@{\hspace{2em}}c}%
			\pruf{%
			\AXC{$\Gamma \vdash f : b/a~~$}
			\AXC{$~~\Delta \vdash e : a$}
			\lab{$/$}
			\BIC{$\Gamma \cdot \Delta \vdash f\,e : b$}
			}
			&
			\pruf{%
			\AXC{$\Delta \vdash e : a~~$}
			\AXC{$~~\Gamma \vdash f : a \backslash b$}
			\lab{$\backslash$}
			\BIC{$\Delta \cdot \Gamma \vdash f\,e : b$}
			}
			&
			\pruf{%
			\AXC{$\Delta \vdash m : \textsf{K}\,(b/a)\,r~~$}
			\AXC{$~~\Gamma \vdash n : \textsf{K}\,a\,r$}
			\lab{$\sslash$}
			\BIC{$\Delta \cdot \Gamma \vdash \bsf{S}\,m\,n : \textsf{K}\,b\,r$}
			}
		\end{array}
		\\\\
		\begin{array}{c@{\hspace{2em}}c}
				\pruf{%
				\AXC{}
				\lab{$\eta$}
				\UIC{$\varepsilon \vdash \eta : a \ra \textsf{M}\,a$}
				}
				&
				\pruf{%
				\AXC{}
				\lab{$\star$}
				\UIC{$\varepsilon \vdash (\star) :  \textsf{M}\,a \ra \textsf{K}\,a\,\textsf{M}\,r$}%
				}
		\end{array}
		\\[-1em]
		\end{array}\]}
		\caption{Continuized CCG with side effects, fixing a monad \ab{\textsf{M},\,\eta,\,\star} and a result type $r$.}%
		\label{fig2}
	\end{figure*}
	%
	Type-theoretic details here. Dylan: I am not entirely sure the type system makes sense. What I'm after: something basically along the lines of \citealt{ShanBarker:2006}, where combinators apply to combinators. Interesting property of that system: they use Lift to allow them only one Scope combinator (i.e.~the thing on the left can always be the functor). Central question: the proper way to relate the continuations mode slashes with the direct mode slashes. The way I did it in my diss appendix was essentially to have a unimodal grammar, but a more elegant solution would be welcome (and important since this is after all a categorial grammar conference!). %
	
	Adding side effects (\citealt{Wadler:1994, Wadler:1995, Shan:2002}): monads. A monad is a triple \ab{\textsf{M},\,\eta,\,\star} of a type constructor \textsf{M}, an injection function $\eta$ of type $a \ra \textsf{M}\,a$ (given any type $a$), and a recipe for sequencing $\star$ of type $\textsf{M}\,a \ra (a \ra \textsf{M}\,b) \ra \textsf{M}\,b$ (given any types $a,b$). %
	
	Monad laws / punting
	\begin{defi}
	\end{defi}
	
	The key to connecting monads with continuations is realizing that the type of $(\star)$ can be rewritten using the continuized type constructor as $\textsf{M}\,a \ra \textsf{K}\,a\,\textsf{M}\,b$%

	Relating monads to continuized grammars: identify \bsf{lift} with $\star$, \bsf{triv} with $\eta$. But \bsf{scope} stays the same.%	
	
	Regular lifting is a theorem, though the types are further specified. For any monad \ab{\textsf{M},\,\eta,\,\star} and any result type $r$:%
	\begin{fact}
		$\Gamma \vdash x : a \Rightarrow \Gamma \vdash \lambda k.\,k\,x : \textsf{K}\,a\,\textsf{M}\,r$%
		%\begin{proof}
		%	Successive applications of $\eta$, $\star$.
		%\end{proof}
	\end{fact}
	\begin{fact}[Monad laws]For any monad \ab{\textsf{M},\,\eta,\,\star}:
		\[\begin{array}{l@{}r@{}l}
			\text{Left Identity: }&\eta\,x \star k &{}= k\,x
			\\
			\text{Right Identity: }&m \star \eta &{}= m
			\\
			\text{Associativity: }&(m \star k) \star c &{}= m \star \lambda x.\,k\,x \star c%
		\end{array}\]
	\end{fact}
%\end{sec}

\section{Finding the dynamic monad}
%\begin{sec}
	Dynamic semantics is (\citealt{Shan:2001}):\footnote{Dynamic treatments following \citealt{GroenendijkStokhof:1990} (e.g.~\citealt{Zimmermann:1991, Dekker:1993, Szabolcsi:2003, Groote:2006}) provide a way for indefinites to extend their binding domain but do not treat indefinites as nondeterministic analogs of proper names.}%
	%\begin{sec}
		State: ability to manipulate the discourse context, i.e.~create discourse referents. %
		Nondeterminism: analogizes indefinites to referential expressions. Treats indefinites as referring expressions, though ones which refer indeterminately.%
	%\end{sec}
	
	The meat of dynamic semantics (\citealt{Heim:1982, Kamp:1981, GroenendijkStokhof:1991, Dekker:1994}): sentences denote relations on sequences. Non-empty relations correspond to truth. Non-functional pairs in the relation correspond to nondeterminism introduced by indefinites (and perhaps disjunction). Conjunction corresponds to relation composition, which pipes the sequences output by the left conjunct to the right conjunct.%
	\[\begin{array}{r@{}l}
		\sv{\text{a linguist}} &{}= \lambda ki.\displaystyle\bigcup_{\mathclap{x\in\,\textsf{ling}}}k\,x\,(i + x)%
		\\
		\sv{\text{a linguist left}} &{}= \lambda i.\,\{i + x : x \in \textsf{ling} \wedge x \in \textsf{left}\}%
	\end{array}\]%

	Two key bits: state modification for introducing drefs, nondeterminism to allow for failure and referring treatment of indefinites. A different perspective on this: treating nondeterminism and state modification as side effects, within a functional programming setting for side effects. %
	
	Monad for state (generalization of monad for environment-sensitivity). Assume that $\gamma$ is the type of ``contexts of evaluation''. For our purposes, we might think of $\gamma$ as inhabited by \emph{sequences of discourse referents}.%
	\begin{defi}[The State monad]\label{state}
		\[\begin{array}[t]
			{@{}l@{}c@{}l@{}}
			\textsf{M}\,a &{}\Coloneqq{} &\gamma \ra a \times \gamma	%
			\\
			\eta\,x &{}\ceq{} &\lambda i.\,\ab{x,\,i}
			\\
			m \star k &{}\ceq{} &\lambda i.\,k\,(m\,i)_0\,(m\,i)_1%
		\end{array}\]
	\end{defi}
	
	Given our identification of $\gamma$ with the set of sequences of discourse referents, a natural operation to suppose as associated with dref introduction is sequence extension (cf.~\citealt{Groote:2006, Unger:2012, Charlow:diss}). These definitions rely on the notion of extending a sequence (e.g., if $i \ceq abcd$, $i+e = abcde$) and retrieving the last, i.e.~most topical, discourse reference (e.g., if $i \ceq abcde$, $i_\top = e$).\footnote{This is an extremely crude measure of topicality, but it will suffice to illustrate the main points.}%
	\begin{defi}[Dref introduction]
		\[m^\rhd \ceq m \star \lambda xi.\,\ab{x,\,i + x}\]
	\end{defi}
	\begin{defi}[Dref retrieval]
		\[\bsf{he} \ceq \lambda i.\,\ab{i_\top,\,i}\]
	\end{defi}
	
	An example, \emph{Al left} (call this \bsf{X}):
	\[(\eta\,\textsf{a})^\rhd\! \star \lambda x.\,\eta\,(\textsf{left}\,x) = \lambda i.\,\ab{\textsf{left}\,\textsf{a},\,i+\textsf{a}}\]%
	
	Pronoun sentence \emph{he was tired} (call this \bsf{Y}):
	\[\bsf{he} \star \lambda x.\,\eta\,(\textsf{tired}\,x) = \lambda i.\,\ab{\textsf{tired}\,i_\top,\,i}\]%
	
	Sequencing the two. The dref introduced by the proper name in the first sentence is accessed by the pronoun in the second. %
	\[\begin{array}{r@{}l}
		\bsf{X} \star \lambda p.\,\bsf{Y}&{}\star \lambda q.\,\eta\,(p \wedge q) %
		\\
		&{}= \lambda i.\,\ab{\textsf{left}\,\textsf{a} \wedge \textsf{tired}\,\textsf{a},\,i+\textsf{a}}%
	\end{array}\]
	
	So we have state modification as a side effect. To say something about indefinites, i.e.~to allow them to refer and introduce drefs nondeterministically, we need to enrich the state monad with nondeterministic side effects. The monad for nondeterminism is the Set monad, given in Definition \ref{set}:%
	\begin{defi}[The Set monad]\label{set}
		\[\begin{array}[t]
			{@{}l@{}c@{}l@{}}
			\textsf{M}\,a &{}\Coloneqq{} &a \ra t%
			\\
			\eta\,x &{}\ceq{} &\{x\}
			\\
			m \star k &{}\ceq{} &\displaystyle\bigcup_{\mathclap{x\in \,m}}k\,x%
		\end{array}\]
	\end{defi} 

	Use StateT to stitch the two together. Given any monad ${\cal M}= \ab{\textsf{L},\,\eta_\textsf{L},\,\star_\textsf{L}}$, StateT is a recipe for building a new monad with which adds State-type functionality to ${\cal M}$:\footnote{Fn. about SetT}. Result also known as the Parser monad. \citealt{HuttonMeijer}%
	\begin{defi}[The StateT monad transformer]\label{statet}
		\[\begin{array}[t]
			{@{}l@{}c@{}l@{}}
			\textsf{M}\,a &{}\Coloneqq{} &\gamma \ra \textsf{L}\,(a \times \gamma)%
			\\
			\eta\,x &{}\ceq{} &\lambda i .\,\eta_\textsf{L}\,\ab{x,\,i}
			\\
			m \star k &{}\ceq{} &\lambda i.\, m\,i\star_\textsf{L} \lambda \pi.\,k\,\pi_0\,\pi_1%
		\end{array}\]
	\end{defi}
	\begin{defi}[The State\_Set monad]\label{stateset}
		\[\begin{array}[t]
			{@{}l@{}c@{}l@{}}
			\textsf{M}\,a &{}\Coloneqq{} &\gamma \ra (a \times \gamma) \ra t%
			\\
			\eta\,x &{}\ceq{} &\lambda i.\left\{\ab{x,\,i}\right\}
			\\
			m \star k &{}\ceq{} &\lambda i.\displaystyle\bigcup_{\mathclap{\pi \in mi}}k\,\pi_0\,\pi_1%
		\end{array}\]
	\end{defi}

	Static lexicon, dynamic lexicon
	
	Modular treatment of binding.%
	\[\begin{array}{ll}
		\text{Previous}:& \bsf{bind}\,m \ceq \lambda k.\,m\,(\lambda x.\,k\,x\,x)%
		\\
		\text{Proposal}:& \bsf{bind}\,m \ceq \lambda k.\,m\,(\lambda xi.\,k\,x\,(i + x))%
	\end{array}\]
%\end{sec}

\section{Examples}
%\begin{sec}	
	Cross-sentential anaphora. Let us assume that \emph{scope islands}, e.g., tensed clauses, need to be evaluated---i.e., lowered (cf.~\citealt{Barker:2002, BarkerShan:2008}). In practice, this means a tensed clause must pass through a stage where it denotes something of type $\textsf{M}\,t$. There are a variety of options for enforcing this syntactically, but here we concentrate on the semantic upshots of forced evaluation.\footnote{In terms of LF, forcing evaluation of a scope island corresponds to disallowing QR out of the scope island.} the indefinite's side effects influence the evaluation of the second clause, even as the indefinite scopes within its clause. %
	\[\begin{array}{r}
		\bsf{so.left} \star \lambda p.\,\bsf{he.tired} \star \lambda q.\,\eta\,(p \wedge q)%
		\\
		= \bsf{so} \star \lambda x.\,\eta\,(\textsf{left}\,x \wedge \textsf{tired}\,x)%
	\end{array}\]
	\begin{figure*}
		%\begin{saveboxes}
		\newsavebox{\partbox}\setbox\partbox=\hbox{\scriptsize\pruf{%
				\AXC{$\bsf{so} : \textsf{M}\,e$}
				\lab{$\star$}
				\UIC{$\lambda k.\,\bsf{so} \star k : \textsf{K}\,e\,\textsf{M}\,t$}%
				\lab{$\rhd$}
				\UIC{$\lambda k.\,\bsf{so}^\rhd\! \star k : \textsf{K}\,e\,\textsf{M}\,t$}%
				\AXC{$\textsf{left} : e \backslash t$}
				\lab{$\uparrow$}
				\UIC{$\lambda k.\,k\,\textsf{left} : \textsf{K}\,(e \backslash t)\,\textsf{M}\,t$}%
				\lab{$\bbslash$}
				\BIC{$\lambda k.\,\bsf{so}^\rhd\! \star \lambda x.\,k\,(\textsf{left}\,x) : \textsf{K}\,t\,\textsf{M}\,t$}%
				\lab{$\downarrow$}
				\UIC{$\bsf{so}^\rhd\! \star \lambda x.\,\eta\,(\textsf{left}\,x) : \textsf{M}\,t$}%
				}}%
		\newsavebox{\partboxx}\setbox\partboxx=\hbox{\scriptsize\pruf{%
				\AXC{$\bsf{he} : \textsf{M}\,e$}
				\lab{$\star$}
				\UIC{$\lambda k.\,\bsf{he} \star k: \textsf{K}\,e\,\textsf{M}\,t$}%
				\AXC{$\textsf{tired} : e \backslash t$}
				\lab{$\uparrow$}
				\UIC{$\lambda k.\,k\,\textsf{tired} : \textsf{K}\,(e \backslash t)\,\textsf{M}\,t$}%
				\lab{$\bbslash$}
				\BIC{$\lambda k.\,\bsf{he} \star \lambda y.\,k\,(\textsf{tired}\,y) : \textsf{K}\,t\,\textsf{M}\,t$}%
				\lab{$\downarrow$}
				\UIC{$\bsf{he} \star \lambda y.\,\eta\,(\textsf{tired}\,y) : \textsf{M}\,t$}}}%
		%\end{saveboxes}
		{\small{\scriptsize\[\begin{array}{c}
			\pruf{%
			\AXC{$\boxed{\usebox\partbox}$}
			\lab{$\star$}
			\UIC{$\lambda k.\,\bsf{so}^\rhd\! \star \lambda x.\,k\,(\textsf{left}\,x) : \textsf{K}\,t\,\textsf{M}\,t$}%
			\AXC{$\textsf{and} : (t \backslash t)/t$}
			\lab{$\uparrow$}
			\UIC{$\lambda k.\,k\,\textsf{and} : \textsf{K}\,((t \backslash t)/t)\,\textsf{M}\,t$}%
			\AXC{$\boxed{\usebox\partboxx}$}
			\lab{$\star$}
			\UIC{$\lambda k.\,\bsf{he} \star \lambda y.\,k\,(\textsf{tired}\,y) : \textsf{K}\,t\,\textsf{M}\,t$}%
			\lab{$\sslash$}
			\BIC{$\lambda k.\,\bsf{he} \star \lambda y.\,k\,(\lambda p.\,p \wedge \textsf{tired}\,y) : \textsf{K}\,(t \backslash t)\,\textsf{M}\,t$}%
			\lab{$\bbslash$}
			\BIC{$\lambda k.\,\bsf{so}^\rhd\! \star \lambda x.\,\bsf{he} \star \lambda y.\,k\,(\textsf{left}\,x \wedge \textsf{tired}\,y) : \textsf{K}\,t\,\textsf{M}\,t$}%
			\lab{$\downarrow$}
			\UIC{$\bsf{so}^\rhd\! \star \lambda x.\,\bsf{he} \star \lambda y.\,\eta\,(\textsf{left}\,x \wedge \textsf{tired}\,y) : \textsf{M}\,t$}%
			\lab{equiv}
			\UIC{$\bsf{so}^\rhd\! \star \lambda x.\,\eta\,(\textsf{left}\,x \wedge \textsf{tired}\,x) : \textsf{M}\,t$}%
			}%
			\\[-1em]
		\end{array}\]}
		\caption{Cross-sentential anaphora: deriving \emph{someone$_i$ left; he$_i$ was tired.}}%
		\label{fig:derivation}}
	\end{figure*}
	
	Negation. Requires there to be no true boolean value returned, tosses out any discourse referents generated in its scope. (Standard). Use to define dynamically closed meanings (e.g.~conditional, universal quantifier, etc.) %
	\[\begin{array}{r@{}l}
		\bsf{not}&{}\ceq \lambda ms.\left\{\ab{\neg\exists \pi \in m\,s.\,\pi_0,\,s}\right\}%
		\\
		\bsf{no.ling}&{}\ceq \lambda k.\,\bsf{not}\,(\bsf{a.ling} \star k)%	
		\\
		\bsf{ev.ling}&{}\ceq \lambda k.\,\bsf{not}\,(\bsf{a.ling} \star \lambda x.\,\bsf{not}\,(k\,x))%
	\end{array}\]%
	
	Compare universals. After ending the derivation at the clause boundary, we're left with a pure computation. The universal's side effects have died on evaluation.%
	\[\eta\,(\forall x.\,\textsf{ling}\,x \Rightarrow \textsf{left}\,x)\]
	
	Donkey anaphora works similarly. Take the following. The restrictor $c$ here acquires a kind of monadic scope, via $\star$, over the nuclear scope $k$. This means any side effects inside $c$ influence the context of evaluation for $k$. However, once $k$ is grabbed, the wide-scoping negation discharges side effects (as is standard in dynamic systems). %	
	\[\bsf{every} \ceq \lambda ck.\,\bsf{not}\,(\bsf{a}\,c \star \lambda x.\,\bsf{not}\,(k\,x))\]%
%\end{sec}

\section{Discussion}
%\begin{sec}
	Compare PLA, where only sentences are imbued with context change potential. Necessary since in PLA and standard dynamic treatments of anaphora, discourse-level content and truth-conditional content are conflated---i.e.~a sentence denotes a non-empty relation on sequences iff the sentence is true. Thus: standard dynamic techniques (DPL, DMG) not reducible to monads. %
	
	Some upshots: no dynamic conjunction, completely standard model theory (cf.~\citealt{Groote:2006}). ``Contexts of evaluation'' are constructed on the fly. Variable-free, directly compositional (\citealt{Jacobson:1999}). %

	Monads as a natural way to extend a continuations-based grammar with tools for dynamic binding and exceptional scope. In the end: you have functional application, plus the functors from whichever monads are implicated in a given language. Effects recognized in the types. %

	There is no need to settle on a single (``the'') grammar. Different and quite varied side effects regimes can be modularly grafted onto a simple applicative (``pure'') core. Lexical entries that would seem incongruous in a flat-footed standard perspective integrate seamlessly in a single grammar. %
	
	Theory extends to scope islands, wide range of exceptional binding configurations \citealt{Charlow:diss}. Extends to pair-list phenomena, functional quantification: \citealt{Bumford:inc}. Crossover, superiority less clear (cf.~\citealt{ShanBarker:2006, BarkerShan:2008}). %
	
	Broader question: how this relates to the idea that continuations can simulate any monad (\citealt{Filinski:1994}). I don't understand this result well enough to say anything (Dylan?). %
%\end{sec}

{\small\bibliography{bqhatevwr}}
\end{document}