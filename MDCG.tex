%\begin{pre}
\documentclass[10pt,twocolumn]{bqhatevwr}
\usepackage{amsmath,mathtools,newtxtext,newtxmath,expex,natbib,tikz-qtree,bookmark,tree-dvips}%
\usepackage[margin=1in]{geometry}
\usepackage{enumitem}
\setlist[itemize]{leftmargin=12pt}
\lingset{exskip=.5em}
\delimitershortfall=-1pt
\bibliographystyle{bqhatevwr}
\author[D.~Bumford \& S.~Charlow]{\spauthor{Dylan Bumford \\ \institute{NYU}} \AND
  \spauthor{Simon Charlow \\ \institute{Rutgers}} \AND
}
\title{Monadic dynamic semantics}
%\end{pre}
\begin{document}
%\maketitle

\section{Overview}
\begin{itemize}	
	\item Any side effects regime can be grafted onto a continuized CCG. Give a general technique for accomplishing this, relate it to the ContT monad transformer (\citealt{Liangetal}).%
	
	\item Yields two combinators. Type-theoretic way to track effects (\citealt{Shan:2005}).%
	
	\item Dynamic semantics is (\citealt{Shan:2001}):\footnote{NB: does not characterize all varieties of dynamic semantics. Dynamic treatments following \citealt{GroenendijkStokhof:1990} (e.g.~\citealt{Zimmermann:1991, Dekker:1993, Szabolcsi:2003, Groote:2006}) provide a way for indefinites to extend their binding domain but do not treat indefinites as nondeterministic analogs of proper names.}%
	\begin{itemize}
		\item State: ability to manipulate the discourse context, i.e.~create discourse referents.%
		\item Nondeterminism: indefinites are 
	\end{itemize}
	
	\item No need to settle on ``the'' grammar. Functional application can live in its bones, and side effects can be grafted on, modularly. Lexical entries that would from a more flat-footed perspective seem completely incongruous can play together nicely. %
	
	\item The perspective \citealt{Barker:2002, ShanBarker:2006, BarkerShan:2014} is in a sense an instantiation of this general perspective, where the underlying monad is the \textsf{Identity} monad.%
	
	\item Monads as a natural way to extend a continuations-based grammar with tools for dynamic binding and exceptional scope. In the end: you have functional application, plus the functors from whichever monads are implicated in a given language. %
	
	\item Other techniques (DPL, DMG) not reducible to monads.
\end{itemize}

\section{Adding side effects to $k$}
\begin{itemize}
	\item Normal continuized grammar:
	\begin{itemize}
		\item Lift
		\item Lower
		\item Continuized functional application
	\end{itemize}

	\item Adding side effects (\citealt{Wadler:1994, Wadler:1995, Shan:2002}): 
	\begin{itemize}
		\item Replace Lift with $\star$
		\item Replace Lower with $\eta$
		\item Keep continuized functional application
	\end{itemize}

	\item Two type constructors:
	\begin{itemize}
		\item Bipartite Cont: 
		\item Unary Monadic: 
	\end{itemize}
	
\end{itemize}

\section{Finding the dynamic monad}
\begin{itemize}
	\item The meat of PLA (\citealt{Dekker:1994}): sentences are relations on stacks. Non-empty relations correspond to truth. Non-functional pairs in the relation correspond to nondeterminism introduced by indefinites (and perhaps disjunction).%
	\[\sv{\text{a linguist}} = \lambda ks.\bigcup_{\mathclap{x\in\,\textsf{ling}}}k\,x\,\widehat{sx}\]%

	\item A different perspective on this. 
	
	\item Monad for nondeterminism:
	\begin{defi}[The Set monad]\label{set}
		\[\begin{array}[t]
			{@{}l@{}c@{}l@{}}
			\textsf{M}\,a &{}\Coloneqq{} &a \ra t%
			\\
			\eta\,x &{}\ceq{} &\{x\}
			\\
			m \star k &{}\ceq{} &\displaystyle\bigcup_{\mathclap{x\in \,m}}k\,x%
		\end{array}\]
	\end{defi}
	
	\item Monad for state (generalization of monad for environment-sensitivity):
	\begin{defi}[The State monad]\label{state}
		\[\begin{array}[t]
			{@{}l@{}c@{}l@{}}
			\textsf{M}\,a &{}\Coloneqq{} &s \ra a \times s%
			\\
			\eta\,x &{}\ceq{} &\lambda s .\,\ab{x,\,s}
			\\
			m \star k &{}\ceq{} &\lambda s.\,k\,(m\,s)_0\,(m\,s)_1%
		\end{array}\]
	\end{defi}
	
	\item Use StateT to stitch the two together\footnote{Fn. about SetT}
	\begin{defi}[The StateT monad transformer]\label{statet}
		\[\begin{array}[t]
			{@{}l@{}c@{}l@{}}
			\textsf{M}\,a &{}\Coloneqq{} &s \ra \textsf{L}\,(a \times s)%
			\\
			\eta\,x &{}\ceq{} &\lambda s .\,\eta_\textsf{L}\,\ab{x,\,s}
			\\
			m \star k &{}\ceq{} &\lambda s.\, m\,s\star_\textsf{L} \lambda \pi.\,k\,\pi_0\,\pi_1%
		\end{array}\]
	\end{defi}
	%
	\begin{defi}[The State\_Set monad]\label{stateset}
		\[\begin{array}[t]
			{@{}l@{}c@{}l@{}}
			\textsf{M}\,a &{}\Coloneqq{} &s \ra (a \times s) \ra t%
			\\
			\eta\,x &{}\ceq{} &\lambda s.\left\{\ab{x,\,s}\right\}
			\\
			m \star k &{}\ceq{} &\lambda s.\displaystyle\bigcup_{\mathclap{\pi\in ms}}k\,\pi_0\,\pi_1%
		\end{array}\]
	\end{defi}
	

	\item Static lexicon, dynamic lexicon
	
	\item Binding, totally modularized
	\[\begin{array}{ll}
		BarkerShan:& \bsf{bind}\,m \ceq \lambda k.\,m\,(\lambda x.\,k\,x\,x)
		\\
		Now:& \bsf{bind}\,m \ceq \lambda k.\,m\,(\lambda xs.\,k\,x\,\widehat{sx})%
	\end{array}\]
	
	\item Summing up: three combinators for ``order-insensitive'' (i.e.~continuized combination). \bsf{unit}, \bsf{run}, \bsf{bind}%
	\[\begin{array}{lllll}
		& \bsf{lift}\,m & M\,\bsf{triv} & \bsf{bind}\,M %& \bsf{scope}\,M\,N%
		\\
		\text{Previous} & \lambda k.\,k\,m & M\,(\lambda x.\,x) & \lambda k.\,m\,(\lambda x.\,k\,x\,x)%
		\\
		\text{Proposal} & \lambda k.\,m \star k & M\,\eta & \lambda k.\,m\,(\lambda xs.\,k\,x\,\widehat{sx})%
	\end{array}\]
\end{itemize}

\section{Examples}
\begin{itemize}
	\item Interesting properties: no dynamic conjunction, completely standard model theory (cf.~\citealt{Groote:2006}). %
\end{itemize}

\citealt{Groote:2001}
\citealt{Charlow:diss}
\citealt{Bumford:inc}

{\small\bibliography{bqhatevwr}}
\end{document}