%\begin{pre}
\documentclass[10pt,twocolumn]{bqhatevwr}
\usepackage{etex,amsmath,mathtools,newtxtext,newtxmath,expex,natbib,tikz-qtree,bookmark,tree-dvips}%
\usepackage[margin=1in]{geometry}
\usepackage{bussproofs}\EnableBpAbbreviations
\newcommand{\pruf}[1]{\def\ScoreOverhang{0pt}\def\defaultHypSeparation{\hskip .25em}#1\DisplayProof}%
\newcommand{\lab}[1]{\RightLabel{\footnotesize #1}}
\usepackage{enumitem}
\setlist[itemize]{leftmargin=12pt}
\lingset{exskip=.5em}
\delimitershortfall=-1pt
\bibliographystyle{bqhatevwr}
\author[D.~Bumford \& S.~Charlow]{\spauthor{Dylan Bumford \\ \institute{NYU}} \AND
  \spauthor{Simon Charlow \\ \institute{Rutgers}} \AND
}
\title{Monadic dynamic semantics}
%\end{pre}
\begin{document}
%\maketitle

\section{Outline}
\begin{itemize}		
	\item Continuized CCGs offer a grammar-wide generalization of scope-taking (``ubiquitous scopal pied piping'') that brings quantifiers and scope-taker more generally into the compositional fold. Three combinators, \bsf{lift}, \bsf{triv}, and \bsf{scope}, form the backbone of the grammar, allowing scope-takers to interact with their linguistic context. In addition, a \bsf{bind} shifter is introduced to facilitate quantificational binding. %
	
	\item Extant CCG treatments of E-type anaphora (i.e.~cross-sentential and donkey anaphora): \citealt{BarkerShan:2008}. Motivation for pursuing another approach. BS approach has not-so-good empirical coverage (\citealt{Charlow:2010}). \citealt{Groote:2006} can be combined with BS regime, but...?%
	
	\item Proposal: replace \bsf{lift} and \bsf{triv} with options that countenance side effects (\citealt{Shan:2005}). Any side effects regime can be grafted onto a continuized CCG, by replacing \bsf{lift} and \bsf{triv} with monadic functors (\citealt{Moggi:1989, Wadler:1992, Wadler:1994, Wadler:1995, Shan:2002}).%
	
	\item We provide a general technique for integrating a monadic approach to side effects with continuations-based approaches to scope in CCG. We relate our approach to the ContT monad transformer (\citealt{Liangetal}). Offers a type-theoretic way to track effects, integrate them into a well-developed CCG framework for scope-taking.%
	
	\item Therefore, these results are of interest both for the categorial grammarian interested in donkey anaphora and scope-taking, as well as more generally. Any side-effects regime a semanticist thinks is motivated can be accommodated along these lines. Further, because of the inherent modularity, adding side effects necessitates neither fiddling with the basic compositional machinery, nor messing with lexical items which don't exploit side effects.%
	
	\item Dynamic semantics is (\citealt{Shan:2001}):\footnote{NB: does not characterize all varieties of dynamic semantics. Dynamic treatments following \citealt{GroenendijkStokhof:1990} (e.g.~\citealt{Zimmermann:1991, Dekker:1993, Szabolcsi:2003, Groote:2006}) provide a way for indefinites to extend their binding domain but do not treat indefinites as nondeterministic analogs of proper names.}%
	\begin{itemize}
		\item State: ability to manipulate the discourse context, i.e.~create discourse referents.%
		\item Nondeterminism: analogizes indefinites to referential expressions. Treats indefinites as referring expressions, though ones which refer indeterminately.%
	\end{itemize}
	
	\item Corollary: there is no need to settle on a single (``the'') grammar. Different and quite varied side effects regimes can be modularly grafted onto a simple applicative (``pure'') core. Lexical entries that would seem incongruous in a flat-footed standard perspective integrate seamlessly in a single grammar. %
	
	\item Monads as a natural way to extend a continuations-based grammar with tools for dynamic binding and exceptional scope. In the end: you have functional application, plus the functors from whichever monads are implicated in a given language. %
	
	\item The standard continuations-based perspective of \citealt{Barker:2002, ShanBarker:2006, BarkerShan:2014} is an instantiation of a more general perspective.%
	
	\item Standard dynamic techniques (DPL, DMG) not reducible to monads. 
	
	\item Broader question: how this relates to the idea that continuations can simulate any monad (\citealt{Filinski:1994}). I don't understand this result well enough to say anything. %
\end{itemize}

\section{Adding side effects to $k$}

\begin{itemize}
	\item Standard continuized grammar:
	\begin{itemize}
		\item \bsf{lift}: $\lambda k.\,k\,x$
		\item \bsf{triv}: $\lambda x.\,x$
		\item \bsf{scope}: $\lambda k.\,m\,(\lambda f.\,n\,(\lambda x.\,k\,(f\,x)))$%
	\end{itemize}
	
	\item To do: insert figure with inference rules.
	\begin{figure*}
		{\small\[\begin{array}{c}
		\begin{array}{cccc}%
			\pruf{%
			\AXC{$\Gamma \vdash f : b/a~~$}
			\AXC{$~~\Delta \vdash e : a$}
			\lab{$/$}
			\BIC{$\Gamma \cdot \Delta \vdash f\,e : b$}
			}
			&
			\pruf{%
			\AXC{$\Delta \vdash e : a~~$}
			\AXC{$~~\Gamma \vdash f : a \backslash b$}
			\lab{$\backslash$}
			\BIC{$\Delta \cdot \Gamma \vdash f\,e : b$}
			}
			&
			\pruf{%
			\AXC{$\Delta \vdash m : \textsf{K}\,(b/a)\,r~~$}
			\AXC{$~~\Gamma \vdash n : \textsf{K}\,a\,r$}
			\lab{$\sslash$}
			\BIC{$\Delta \cdot \Gamma \vdash \bsf{S}_/ m\,n : \textsf{K}\,b\,r$}
			}
			&
			\pruf{%
			\AXC{$\Delta \vdash m : \textsf{K}\,a\,r~~$}
			\AXC{$~~\Gamma \vdash n : \textsf{K}\,(a \backslash b)\,r$}
			\lab{$\bbslash$}
			\BIC{$\Delta \cdot \Gamma \vdash \bsf{S}_\backslash m\,n : \textsf{K}\,b\,r$}%
			}
		\end{array}
		\\\\
		\begin{array}{cc}
			\pruf{%
			\AXC{$\Gamma \vdash e : a$}
			\lab{$\uparrow$}
			\UIC{$\Gamma \vdash \lambda k.\,k\,e : \textsf{K}\,a\,r$}%
			}
			&
			\pruf{%
			\AXC{$\Gamma \vdash m : \textsf{K}\,r\,r$}
			\lab{$\downarrow$}
			\UIC{$\Gamma \vdash m\,(\lambda x.\,x) : r$}%
			}
		\end{array}
		\\[-1em]
		\end{array}\]}
		\caption{Partial multimodal continuized grammar, no side effects.}
		\label{fig1}
	\end{figure*}
	\begin{figure*}
		{\small\[\begin{array}{c}
		\begin{array}{cccc}%
			\pruf{%
			\AXC{$\Gamma \vdash f : b/a~~$}
			\AXC{$~~\Delta \vdash e : a$}
			\lab{$/$}
			\BIC{$\Gamma \cdot \Delta \vdash f\,e : b$}
			}
			&
			\pruf{%
			\AXC{$\Delta \vdash e : a~~$}
			\AXC{$~~\Gamma \vdash f : a \backslash b$}
			\lab{$\backslash$}
			\BIC{$\Delta \cdot \Gamma \vdash f\,e : b$}
			}
			&
			\pruf{%
			\AXC{$\Delta \vdash m : \textsf{K}\,(b/a)\,r~~$}
			\AXC{$~~\Gamma \vdash n : \textsf{K}\,a\,r$}
			\lab{$\sslash$}
			\BIC{$\Delta \cdot \Gamma \vdash \bsf{S}_/ m\,n : \textsf{K}\,b\,r$}
			}
			&
			\pruf{%
			\AXC{$\Delta \vdash m : \textsf{K}\,a\,r~~$}
			\AXC{$~~\Gamma \vdash n : \textsf{K}\,(a \backslash b)\,r$}
			\lab{$\bbslash$}
			\BIC{$\Delta \cdot \Gamma \vdash \bsf{S}_\backslash m\,n : \textsf{K}\,b\,r$}%
			}
		\end{array}
		\\\\
		\begin{array}{cc}
			\pruf{%
			\AXC{$\Gamma \vdash e : \textsf{M}\,a$}
			\lab{$\uparrow$}
			\UIC{$\Gamma \vdash \lambda k.\,e \star k : \textsf{K}\,a\,\textsf{M}\,r$}%
			}
			&
			\pruf{%
			\AXC{$\Gamma \vdash m : \textsf{K}\,r\,\textsf{M}\,r$}
			\lab{$\downarrow$}
			\UIC{$\Gamma \vdash m\,\eta : \textsf{M}\,r$}%
			}
		\end{array}
		\\[-1em]
		\end{array}\]}
		\caption{Partial multimodal continuized grammar, with side effects.}
		\label{fig2}
	\end{figure*}
	
	\item Type-theoretic details here
	
	\item Adding side effects (\citealt{Wadler:1994, Wadler:1995, Shan:2002}): monads%
	
	\item Monad laws / punting
	
	\item Relating monads to continuized grammars:
	\begin{itemize}
		\item Replace \bsf{lift} with $\star$
		\item Replace \bsf{triv} with $\eta$
		\item \bsf{scope} stays the same
	\end{itemize}

	\item Two type constructors:
	\begin{itemize}
		\item Bipartite Cont: $\textsf{K}\,a\,b \Coloneqq (a \ra b) \ra b$
		\item Unary Monadic: 
	\end{itemize}
	
\end{itemize}

\section{Finding the dynamic monad}
\begin{itemize}
	\item The meat of PLA (\citealt{Dekker:1994}): sentences are relations on stacki. Non-empty relations correspond to truth. Non-functional pairs in the relation correspond to nondeterminism introduced by indefinites (and perhaps disjunction).%
	\[\sv{\text{a linguist}} = \lambda ki.\bigcup_{\mathclap{x\in\,\textsf{ling}}}k\,x\,\widehat{ix}\]%

	\item A different perspective on this: treating nondeterminism and state modification as side effects, within a functional programming setting for side effects. %
	
	\item Monad for nondeterminism:
	\begin{defi}[The Set monad]\label{set}
		\[\begin{array}[t]
			{@{}l@{}c@{}l@{}}
			\textsf{M}\,a &{}\Coloneqq{} &a \ra t%
			\\
			\eta\,x &{}\ceq{} &\{x\}
			\\
			m \star k &{}\ceq{} &\displaystyle\bigcup_{\mathclap{x\in \,m}}k\,x%
		\end{array}\]
	\end{defi}
	
	\item Monad for state (generalization of monad for environment-sensitivity). Assume that $\gamma$ is the type of ``contexts of evaluation''. For our purposes, we might think of $\gamma$ as inhabited by \emph{sequences of discourse referents}.%
	\begin{defi}[The State monad]\label{state}
		\[\begin{array}[t]
			{@{}l@{}c@{}l@{}}
			\textsf{M}\,a &{}\Coloneqq{} &\gamma \ra a \times \gamma	%
			\\
			\eta\,x &{}\ceq{} &\lambda i.\,\ab{x,\,i}
			\\
			m \star k &{}\ceq{} &\lambda i.\,k\,(m\,i)_0\,(m\,i)_1%
		\end{array}\]
	\end{defi}
	
	\item Given our identification of $\gamma$ with the set of sequences of discourse referents, a natural operation to suppose as associated with dref introduction is sequence extension (cf.~\citealt{Groote:2006, Unger:2012}): %
	\begin{defi}[Sequence extension]
		\[m^\rhd \ceq m \star \lambda xi.\,\ab{x,\,\widehat{ix}}\]
	\end{defi}
	
	\item An example:
	\[(\eta\,\textsf{a})^\rhd\! \star \lambda x.\,\eta\,(\textsf{left}\,x) = \lambda i.\,\ab{\textsf{left}\,\textsf{a},\,\widehat{i\textsf{a}}}\]%
	
	\item Use StateT to stitch the two together. Given any monad ${\cal M}= \ab{\textsf{L},\,\eta_\textsf{L},\,\star_\textsf{L}}$, StateT is a recipe for building a new monad with which adds State-type functionality to ${\cal M}$:\footnote{Fn. about SetT}%
	\begin{defi}[The StateT monad transformer]\label{statet}
		\[\begin{array}[t]
			{@{}l@{}c@{}l@{}}
			\textsf{M}\,a &{}\Coloneqq{} &\gamma \ra \textsf{L}\,(a \times \gamma)%
			\\
			\eta\,x &{}\ceq{} &\lambda i .\,\eta_\textsf{L}\,\ab{x,\,i}
			\\
			m \star k &{}\ceq{} &\lambda i.\, m\,i\star_\textsf{L} \lambda \pi.\,k\,\pi_0\,\pi_1%
		\end{array}\]
	\end{defi}
	%
	\begin{defi}[The State\_Set monad]\label{stateset}
		\[\begin{array}[t]
			{@{}l@{}c@{}l@{}}
			\textsf{M}\,a &{}\Coloneqq{} &\gamma \ra (a \times \gamma) \ra t%
			\\
			\eta\,x &{}\ceq{} &\lambda i.\left\{\ab{x,\,i}\right\}
			\\
			m \star k &{}\ceq{} &\lambda i.\displaystyle\bigcup_{\mathclap{\pi \in mi}}k\,\pi_0\,\pi_1%
		\end{array}\]
	\end{defi}
	

	\item Static lexicon, dynamic lexicon
	
	\item Modular treatment of binding.%
	\[\begin{array}{ll}
		\text{Previous}:& \bsf{bind}\,m \ceq \lambda k.\,m\,(\lambda x.\,k\,x\,x)%
		\\
		\text{Proposal}:& \bsf{bind}\,m \ceq \lambda k.\,m\,(\lambda xi.\,k\,x\,\widehat{ix})%
	\end{array}\]
	
	\item Summing up: three combinators for ``order-insensitive'' (i.e.~continuized combination). \bsf{unit}, \bsf{run}, \bsf{bind}%
	\[\begin{array}{lllll}
		& \bsf{lift}\,m & M\,\bsf{triv} & \bsf{bind}\,M %& \bsf{scope}\,M\,N%
		\\
		\text{Previous} & \lambda k.\,k\,m & M\,(\lambda x.\,x) & \lambda k.\,m\,(\lambda x.\,k\,x\,x)%
		\\
		\text{Proposal} & \lambda k.\,m \star k & M\,\eta & \lambda k.\,m\,(\lambda xi.\,k\,x\,\widehat{ix})%
	\end{array}\]
\end{itemize}

\section{Examples}
\begin{itemize}
	\item Some upshots: no dynamic conjunction, completely standard model theory (cf.~\citealt{Groote:2006}). ``Contexts of evaluation'' are constructed on the fly. %
	
	\item Cross-sentential anaphora: the indefinite's side effects influence the evaluation of the second clause, even as the indefinite scopes within its clause. %
	\[\begin{array}{r}
		\bsf{a.man.left} \star \lambda p.\,\bsf{he.tired} \star \lambda q.\,\eta\,(p \wedge q)%
		\\
		= \bsf{a.man} \star \lambda x.\,\eta\,(\textsf{left}\,x \wedge \textsf{tired}\,x)%
	\end{array}\]
	
	\item Compare universals. After ending the derivation at the clause boundary, we're left with a pure computation. The universal's side effects have died on evaluation.%
	\[\eta\,(\forall x.\,\textsf{ling}\,x \Rightarrow \textsf{left}\,x)\]
	
	\item Donkey anaphora works similarly. Take the following. The restrictor $c$ here acquires a kind of monadic scope, via $\star$, over the nuclear scope $k$. This means any side effects inside $c$ influence the context of evaluation for $k$. However, once $k$ is grabbed, the wide-scoping negation discharges side effects (as is standard in dynamic systems). %
	\[\sv{\text{every}} \ceq \lambda ck.\,\bsf{not}\,(\bsf{a}\,c \star \lambda x.\,\bsf{not}\,(k\,x))\]%
	
	\item Islands: a clause must denote a $\textsf{M}\,t$
\end{itemize}

\citealt{Groote:2001}
\citealt{Charlow:diss}
\citealt{Bumford:inc}

{\small\bibliography{bqhatevwr}}
\end{document}