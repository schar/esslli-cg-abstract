%\begin{pre}
\documentclass[10pt]{sp_2000}
\usepackage{amsmath,mathtools,newtxtext,newtxmath,expex,natbib,tikz-qtree,bookmark,tree-dvips}%
\lingset{exskip=.5em}
\newcommand{\posscite}[1]{\citeauthor{#1}'s \citeyear{#1}}
\usepackage[normalem]{ulem}
\newcommand{\ra}{\ensuremath{\rightarrow}} 
\newcommand{\bsf}[1]{\textbf{\textsf{#1}}}
\newcommand{\ceq}{\ensuremath{\coloneqq}}
\newcommand{\ab}[1]{\ensuremath{\langle #1 \rangle}}
\newcommand{\foc}[1]{\ensuremath{\textsc{#1}_{\textsc{f}}}}
\newcommand{\rex}[1]{(\ref{#1})}
\usepackage{combelow}
\newcommand*\phantomas[3][c]{%
\ifmmode
\makebox[\widthof{$#2$}][#1]{$#3$}%
\else
\makebox[\widthof{#2}][#1]{#3}%
\fi
}
\bibliographystyle{sp-new}
\author[D.~Bumford \& S.~Charlow]{\spauthor{Dylan Bumford \\ \institute{NYU}} \AND
  \spauthor{Simon Charlow \\ \institute{Rutgers}} \AND
}
\title{Monadic dynamic semantics}
\delimitershortfall=-1pt
%\end{pre}

\begin{document}
%\maketitle

\section{Overview}
\begin{itemize}	
	\item Any side effects regime can be grafted onto a continuized CCG. Give a general technique for accomplishing this, relate it to the ContT monad transformer (\citealt{Liangetal}).%
	
	\item Yields two combinators. Type-theoretic way to track effects (\citealt{Shan:2005}).%
	
	\item Dynamic semantics is (\citealt{Shan:2001}):\footnote{NB: does not characterize all varieties of dynamic semantics. Dynamic treatments following \citealt{GroenendijkStokhof:1990} (e.g.~\citealt{Zimmermann:1991, Szabolcsi:2003, Groote:2006}) provide a way for indefinites to extend their binding domain but do not treat indefinites as nondeterministic analogs of proper names.}%
	\begin{itemize}
		\item State: ability to manipulate the discourse context, i.e.~create discourse referents.%
		\item Nondeterminism: indefinites are 
	\end{itemize}
	
	\item No need to settle on ``the'' grammar. Functional application can live in its bones, and side effects can be grafted on, modularly. Lexical entries that would from a more flat-footed perspective seem completely incongruous can play together nicely. %
	
	\item The perspective \citealt{Barker:2002, ShanBarker:2006, BarkerShan:2014} is in a sense an instantiation of this general perspective, where the underlying monad is the \textsf{Identity} monad.%
	
	\item Monads as a natural way to extend a continuations-based grammar with tools for dynamic binding and exceptional scope. In the end: you have functional application, plus the functors from whichever monads are implicated in a given language. %
	
	\item Other techniques (DPL, DMG) not reducible to monads.
\end{itemize}

\section{Adding side effects to $\kappa$}
\begin{itemize}
	\item Normal continuized grammar:
	\begin{itemize}
		\item Lift
		\item Lower
		\item Continuized functional application
	\end{itemize}

	\item Adding side effects (\citealt{Wadler:1994, Wadler:1995, Shan:2002}): 
	\begin{itemize}
		\item Replace Lift with $\star$
		\item Replace Lower with $\eta$
		\item Keep continuized functional application
	\end{itemize}

	\item Two type constructors:
	\begin{itemize}
		\item Bipartite Cont: 
		\item Unary Monadic: 
	\end{itemize}
	
\end{itemize}


\section{Finding the dynamic monad}

\begin{itemize}
	
	\item A PLA-style system.
	
	\item Monad for state
	
	\item Monad for nondeterminism
	
	\item Use StateT to stitch the two together

	\item Static lexicon, dynamic lexicon
	
	\item Examples
	
	\item Interesting properties: no dynamic conjunction, completely standard model theory (cf.~\citealt{Groote:2006}). %
	
	\item Binding, totally modularized
	\begin{itemize}
		\item BarkerShan:
		\item $\rhd$
	\end{itemize}
	
\end{itemize}

\section{}

\citealt{Groote:2001}
\citealt{Charlow:diss}

{\small\bibliography{diss}}

\end{document}